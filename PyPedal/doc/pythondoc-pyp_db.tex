% This file was converted from HTML to LaTeX with
% Tomasz Wegrzanowski's <maniek@beer.com> gnuhtml2latex program
% Version : 0.1
\documentclass{article}
\begin{document}

\section*{The pyp\_db Module}
\par pyp\_db contains a set of procedures for ...
\begin{description}
\item[\textbf{createPedigreeDatabase(dbname='pypedal')} ⇒ integer [\#]
]
\par createPedigreeDatabase() creates a new database in SQLite.
\begin{description}
\item[\textit{dbname}
]
The name of the database to create.
\item[Returns:
]
A 1 on successful database creation, a 0 otherwise.
\end{description}\\

\item[\textbf{createPedigreeTable(curs, tablename='example')} ⇒ integer [\#]
]
\par createPedigreeDatabase() creates a new pedigree table in a SQLite
database.
\begin{description}
\item[\textit{tablename}
]
The name of the table to create.
\item[Returns:
]
A 1 on successful table creation, a 0 otherwise.
\end{description}\\

\item[\textbf{databaseQuery(sql, curs=0, dbname='pypedal')} ⇒ string [\#]
]
\par databaseQuery() executes an SQLite query.  This is a wrapper function
used by the reporting functions that need to fetch data from SQLite.
I wrote it so that any changes that need to be made in the way PyPedal
talks to SQLite will only need to be changed in one place.
\begin{description}
\item[\textit{sql}
]
A string containing an SQL query.
\item[\textit{\_curs}
]
An [optional] SQLite cursor.
\item[\textit{dbname}
]
The database into which the pedigree will be loaded.
\item[Returns:
]
The results of the query, or 0 if no resultset.
\end{description}\\

\item[\textbf{getCursor(dbname='pypedal')} ⇒ cursor [\#]
]
\par getCursor() creates a database connection and returns a
cursor on success or a 0 on failure.  It is very useful for
non-trivial queries because it creates SQLite aggrefates
before returning the cursor.  The reporting routines in pyp\_reports
make heavy use of getCursor().
\begin{description}
\item[\textit{dbname}
]
The database into which the pedigree will be loaded.
\item[Returns:
]
An SQLite cursor if the database exists, a 0 otherwise.
\end{description}\\

\item[\textbf{getCursorSQA(dbtype='sqlite', dbname='pypedal', dbuser='', dbpasswd='', dbserver='', dbdebug=False)} ⇒ engine [\#]
]
\par getCursorSQA() creates a database connection and returns a
cursor on success or a 0 on failure. It uses SQLAlchemy.
The reporting routines in pyp\_reports make heavy use of
getCursor().
\begin{description}
\item[\textit{dbtype}
]
Type of databsae to which to connect (any valid SQLAlchemy dbtype).
\item[\textit{dbname}
]
The database into which the pedigree will be loaded.
\item[\textit{dbuser}
]
Usernam to use when connecting to the database.
\item[\textit{dbpasswd}
]
Password to use when connecting to the database.
\item[\textit{dbserver}
]
Name of the server hosting the database.
\item[\textit{dbdebug}
]
TUrns debugging output on and off (True|False).
\item[Returns:
]
An instance of an SQLAlchemy database engine if the database exists, a 0 otherwise.
\end{description}\\

\item[\textbf{loadPedigreeTable(pedobj)} ⇒ integer [\#]
]
\par loadPedigreeTable() takes a PyPedal pedigree object and loads
the animal records in that pedigree into an SQLite table.
\begin{description}
\item[\textit{pedobj}
]
A PyPedal pedigree object.
\item[\textit{dbname}
]
The database into which the pedigree will be loaded.
\item[\textit{tablename}
]
The table into which the pedigree will be loaded.
\item[Returns:
]
A 1 on successful table load, a 0 otherwise.
\end{description}\\

\item[\textbf{PypMean()} (class) [\#]
]
\par PypMean is a user-defined aggregate for SQLite for returning means from queries.
\par For more information about this class, see \textit{The PypMean Class}.

\item[\textbf{PypSSD()} (class) [\#]
]
\par PypSSD is a user-defined aggregate for SQLite for returning sample standard deviations
from queries.
\par For more information about this class, see \textit{The PypSSD Class}.

\item[\textbf{PypSum()} (class) [\#]
]
\par PypSum is a user-defined aggregate for SQLite for returning sums from queries.
\par For more information about this class, see \textit{The PypSum Class}.

\item[\textbf{PypSVar()} (class) [\#]
]
\par PypSVar is a user-defined aggregate for SQLite for returning sample variances
from queries.
\par For more information about this class, see \textit{The PypSVar Class}.

\item[\textbf{tableCountRows(dbname='pypedal', tablename='example')} ⇒ integer [\#]
]
\par tableCountRows() returns the number of rows in a table.
\begin{description}
\item[\textit{dbname}
]
The database into which the pedigree will be loaded.
\item[\textit{tablename}
]
The table into which the pedigree will be loaded.
\item[Returns:
]
The number of rows in the table 1 or 0.
\end{description}\\

\item[\textbf{tableDropRows(dbname='pypedal', tablename='example')} ⇒ integer [\#]
]
\par tableDropRows() drops all of the data from an existing table.
\begin{description}
\item[\textit{dbname}
]
The database from which data will be dropped.
\item[\textit{tablename}
]
The table from which data will be dropped.
\item[Returns:
]
A 1 if the data were dropped, a 0 otherwise.
\end{description}\\

\item[\textbf{tableDropTable(dbname='pypedal', tablename='example')} ⇒ integer [\#]
]
\par tableDropTable() drops a table from the database.
\begin{description}
\item[\textit{dbname}
]
The database from which the table will be dropped.
\item[\textit{tablename}
]
The table which will be dropped.
\item[Returns:
]
1
\end{description}\\

\item[\textbf{tableExists(dbname='pypedal', tablename='example')} ⇒ integer [\#]
]
\par tableExists() queries the sqlite\_master view in an SQLite database to
determine whether or not a table exists.
\begin{description}
\item[\textit{dbname}
]
The database into which the pedigree will be loaded.
\item[\textit{tablename}
]
The table into which the pedigree will be loaded.
\item[Returns:
]
A 1 if the table exists, a 0 otherwise.
\end{description}\\

\item[\textbf{tableExistsSQA(db, dbtable)} ⇒ integer [\#]
]
\par tableExistsSQA() queries the sqlite\_master view in an SQLite database to
determine whether or not a table exists.
\begin{description}
\item[\textit{db}
]
An instance of an SQLAlchemy database engine.
\item[\textit{dbtable}
]
Name of the table whose existence we're testing.
\item[Returns:
]
A 1 if the table exists, a 0 otherwise.
\end{description}\\

\end{description}
\subsection*{The PypMean Class}
\begin{description}
\item[\textbf{PypMean()} (class) [\#]
]
\par PypMean is a user-defined aggregate for SQLite for returning means from queries.

\end{description}
\subsection*{The PypSSD Class}
\begin{description}
\item[\textbf{PypSSD()} (class) [\#]
]
\par PypSSD is a user-defined aggregate for SQLite for returning sample standard deviations
from queries.

\end{description}
\subsection*{The PypSum Class}
\begin{description}
\item[\textbf{PypSum()} (class) [\#]
]
\par PypSum is a user-defined aggregate for SQLite for returning sums from queries.

\end{description}
\subsection*{The PypSVar Class}
\begin{description}
\item[\textbf{PypSVar()} (class) [\#]
]
\par PypSVar is a user-defined aggregate for SQLite for returning sample variances
from queries.

\end{description}
\end{document}
