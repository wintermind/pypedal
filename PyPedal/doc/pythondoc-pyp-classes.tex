\par pyp\_classes contains two base classes that are used by PyPedal, the Animal() class
and the Pedigree() class.  What most PyPedal routines recognize as a pedigree is
actually just a Python list of Animal() objects.  An instance of a Pedigree() object
is a collection of METADATA about a list of Animals().  I know that this is confusing,
and it is going to change by the time that PyPedal 2.0.0 final is released.
\begin{description}
\item[\textbf{Animal(animalID, sireID, damID, gen='0', by=1900, sex='u', fa=0., name='u', alleles=['', ''], breed='u', age=-999, alive=-999)} (class) [\#]
]
\par The Animal() class is holds animals records read from a pedigree file.
\par For more information about this class, see \textit{The Animal Class}.

\item[\textbf{Pedigree(myped, inputfile, name, pedcode='asd', reord=0, renum=0, debug=0)} (class) [\#]
]
\par The Pedigree() class stores metadata about pedigrees.
\par For more information about this class, see \textit{The Pedigree Class}.

\end{description}
\subsection*{The Animal Class}
\begin{description}
\item[\textbf{Animal(animalID, sireID, damID, gen='0', by=1900, sex='u', fa=0., name='u', alleles=['', ''], breed='u', age=-999, alive=-999)} (class) [\#]
]
\par The Animal() class is holds animals records read from a pedigree file.

\item[\textbf{\_\_init\_\_(animalID, sireID, damID, gen='0', by=1900, sex='u', fa=0., name='u', alleles=['', ''], breed='u', age=-999, alive=-999)} ⇒ object [\#]
]
\par \_\_init\_\_() initializes an Animal() object.
\begin{description}
\item[\textit{self}
]
Reference to the current Animal() object
\item[\textit{animalID}
]
Animal ID number
\item[\textit{sireID}
]
Sire ID number
\item[\textit{damID}
]
Dam ID number
\item[\textit{gen}
]
Generation to which the animal belongs
\item[\textit{by}
]
Birthyear of the animal
\item[\textit{sex}
]
Sex of the animal (m|f|u)
\item[\textit{fa}
]
Coefficient of inbreeding of the animal
\item[\textit{name}
]
Name of animal
\item[\textit{alleles}
]
A two-element array of strings, which represent allelotypes.
\item[\textit{breed}
]
Breed of animal
\item[\textit{age}
]
Age of animal
\item[\textit{alive}
]
Status of animal (alive or dead)
\item[Returns:
]
An instance of an Animal() object populated with data
\end{description}\\

\item[\textbf{pad\_id()} ⇒ integer [\#]
]
\par pad\_id() takes an Animal ID, pads it to fifteen digits, and prepends the birthyear
(or 1950 if the birth year is unknown).  The order of elements is: birthyear, animalID,
count of zeros, zeros.
\begin{description}
\item[\textit{self}
]
Reference to the current Animal() object
\item[Returns:
]
A padded ID number that is supposed to be unique across animals
\end{description}\\

\item[\textbf{printme()} [\#]
]
\par printme() prints a summary of the data stored in the Animal() object.
\begin{description}
\item[\textit{self}
]
Reference to the current Animal() object
\end{description}\\

\item[\textbf{stringme()} [\#]
]
\par stringme() returns a summary of the data stored in the Animal() object
as a string.
\begin{description}
\item[\textit{self}
]
Reference to the current Animal() object
\end{description}\\

\item[\textbf{trap()} [\#]
]
\par trap() checks for common errors in Animal() objects
\begin{description}
\item[\textit{self}
]
Reference to the current Animal() object
\end{description}\\

\end{description}
\subsection*{The Pedigree Class}
\begin{description}
\item[\textbf{Pedigree(myped, inputfile, name, pedcode='asd', reord=0, renum=0, debug=0)} (class) [\#]
]
\par The Pedigree() class stores metadata about pedigrees.  Hopefully this will help improve performance in some procedures,
as well as provide some useful summary data.

\item[\textbf{\_\_init\_\_(myped, inputfile, name, pedcode='asd', reord=0, renum=0, debug=0)} ⇒ object [\#]
]
\par \_\_init\_\_() initializes a Pedigree metata object.
\begin{description}
\item[\textit{self}
]
Reference to the current Pedigree() object
\item[\textit{myped}
]
A PyPedal pedigree
\item[\textit{inputfile}
]
The name of the file from which the pedigree was loaded
\item[\textit{name}
]
The name assigned to the PyPedal pedigree
\item[\textit{pedcode}
]
The format code for the PyPedal pedigree
\item[\textit{reord}
]
Flag indicating whether or not the pedigree is reordered (0|1)
\item[\textit{renum}
]
Flag indicating whether or not the pedigree is renumbered (0|1)
\item[Returns:
]
An instance of a Pedigree() object populated with data
\end{description}\\

\item[\textbf{fileme()} [\#]
]
\par fileme() writes the metada stored in the Pedigree() object to disc.
\begin{description}
\item[\textit{self}
]
Reference to the current Pedigree() object
\end{description}\\

\item[\textbf{nud()} ⇒ integer-and-list [\#]
]
\par nud() returns the number of unique dams in the pedigree along with a list of the dams
\begin{description}
\item[\textit{self}
]
Reference to the current Pedigree() object
\item[Returns:
]
The number of unique dams in the pedigree and a list of those dams
\end{description}\\

\item[\textbf{nuf()} ⇒ integer-and-list [\#]
]
\par nuf() returns the number of unique founders in the pedigree along with a list of the founders
\begin{description}
\item[\textit{self}
]
Reference to the current Pedigree() object
\item[Returns:
]
The number of unique founders in the pedigree and a list of those founders
\end{description}\\

\item[\textbf{nug()} ⇒ integer-and-list [\#]
]
\par nug() returns the number of unique generations in the pedigree along with a list of the generations
\begin{description}
\item[\textit{self}
]
Reference to the current Pedigree() object
\item[Returns:
]
The number of unique generations in the pedigree and a list of those generations
\end{description}\\

\item[\textbf{nus()} ⇒ integer-and-list [\#]
]
\par nus() returns the number of unique sires in the pedigree along with a list of the sires
\begin{description}
\item[\textit{self}
]
Reference to the current Pedigree() object
\item[Returns:
]
The number of unique sires in the pedigree and a list of those sires
\end{description}\\

\item[\textbf{nuy()} ⇒ integer-and-list [\#]
]
\par nuy() returns the number of unique birthyears in the pedigree along with a list of the birthyears
\begin{description}
\item[\textit{self}
]
Reference to the current Pedigree() object
\item[Returns:
]
The number of unique birthyears in the pedigree and a list of those birthyears
\end{description}\\

\item[\textbf{printme()} [\#]
]
\par printme() prints a summary of the metadata stored in the Pedigree() object.
\begin{description}
\item[\textit{self}
]
Reference to the current Pedigree() object
\end{description}\\

\item[\textbf{stringme()} [\#]
]
\par stringme() returns a summary of the metadata stored in the pedigree as
a string.
\begin{description}
\item[\textit{self}
]
Reference to the current Pedigree() object
\end{description}\\

\end{description}
