% This file was converted from HTML to LaTeX with
% Tomasz Wegrzanowski's <maniek@beer.com> gnuhtml2latex program
% Version : 0.1
\documentclass{article}
\begin{document}

\section*{The pyp\_io Module}
\par pyp\_io contains several procedures for writing structures to and reading them from
disc (e.g. using pickle() to store and retrieve A and A-inverse).  It also includes a set
of functions used to render strings as HTML or plaintext for use in generating output
files.
\begin{description}
\item[\textbf{a\_inverse\_from\_file(inputfile)} ⇒ matrix [\#]
]
\par a\_inverse\_from\_file() uses the Python pickle system for persistent objects to read the inverse of
a relationship matrix from a file.
\begin{description}
\item[\textit{inputfile}
]
The name of the input file.
\item[Returns:
]
The inverse of a numerator relationship matrix.
\end{description}\\

\item[\textbf{a\_inverse\_to\_file(pedobj, ainv='')} ⇒ integer [\#]
]
\par a\_inverse\_to\_file() uses the Python pickle system for persistent objects to write the
inverse of a relationship matrix to a file.
\begin{description}
\item[\textit{pedobj}
]
A PyPedal pedigree object.
\item[\textit{filetag}
]
A descriptor prepended to output file names.
\item[Returns:
]
True (1) on success, false (0) on failure
\end{description}\\

\item[\textbf{dissertation\_pedigree\_to\_file(pedobj)} ⇒ integer [\#]
]
\par dissertation\_pedigree\_to\_file() takes a pedigree in 'asdxfg' format and writes is to a file.
\begin{description}
\item[\textit{pedobj}
]
A PyPedal pedigree object.
\item[Returns:
]
True (1) on success, false (0) on failure
\end{description}\\

\item[\textbf{dissertation\_pedigree\_to\_pedig\_format(pedobj)} ⇒ integer [\#]
]
\par dissertation\_pedigree\_to\_pedig\_format() takes a pedigree in 'asdbxfg' format, formats it into
the form used by Didier Boichard's 'pedig' suite of programs, and writes it to a file.
\begin{description}
\item[\textit{pedobj}
]
A PyPedal pedigree object.
\item[Returns:
]
True (1) on success, false (0) on failure
\end{description}\\

\item[\textbf{dissertation\_pedigree\_to\_pedig\_format\_mask(pedobj)} ⇒ integer [\#]
]
\par dissertation\_pedigree\_to\_pedig\_format\_mask() Takes a pedigree in 'asdbxfg' format,
formats it into the form used by Didier Boichard's 'pedig' suite of programs, and
writes it to a file.  THIS FUNCTION MASKS THE GENERATION ID WITH A FAKE BIRTH YEAR
AND WRITES THE FAKE BIRTH YEAR TO THE FILE INSTEAD OF THE TRUE BIRTH YEAR.  THIS IS
AN ATTEMPT TO FOOL PEDIG TO GET f\_e, f\_a et al. BY GENERATION.
\begin{description}
\item[\textit{pedobj}
]
A PyPedal pedigree object.
\item[Returns:
]
True (1) on success, false (0) on failure
\end{description}\\

\item[\textbf{dissertation\_pedigree\_to\_pedig\_interest\_format(pedobj)} ⇒ integer [\#]
]
\par dissertation\_pedigree\_to\_pedig\_interest\_format() takes a pedigree in 'asdbxfg' format,
formats it into the form used by Didier Boichard's parente program for the studied
individuals file.
\begin{description}
\item[\textit{pedobj}
]
A PyPedal pedigree object.
\item[Returns:
]
True (1) on success, false (0) on failure
\end{description}\\

\item[\textbf{load\_from\_gedcom(infilename, messages='verbose', standalone=1, missing\_sex='u', missing\_parent=0, missing\_name='Unknown Name', missing\_byear='0001', debug=0)} ⇒ integer [\#]
]
\par load\_from\_gedcom() reads and parses pedigree data that conform to
a subset of the GEDCOM 5.5 specification. Not all valid GEDCOM
are supported; unsupported tags are ignored.
\begin{description}
\item[\textit{infilename}
]
The file to which the matrix should be written.
\item[\textit{messages}
]
Controls output to the screen
\item[\textit{standalone}
]
Uses logging if called by a NewPedigree method
\item[\textit{missing\_sex}
]
Value assigned to an animal with unknown sex
\item[\textit{missing\_parent}
]
Value assigned to unknown parents
\item[\textit{missing\_name}
]
Name assigned by default
\item[\textit{missing\_byear}
]
VAlue assigned to unknown birth years
\item[\textit{debug}
]
Flag turning debugging messages on (1) and off (0)
\item[Returns:
]
A save status indicator (0: failed, 1: success).
\end{description}\\

\item[\textbf{pickle\_pedigree(pedobj, filename='')} ⇒ integer [\#]
]
\par pickle\_pedigree() pickles a pedigree.
\begin{description}
\item[\textit{pedobj}
]
An instance of a PyPedal pedigree object.
\item[\textit{filename}
]
The name of the file to which the pedigree object should be pickled (optional).
\item[Returns:
]
A 1 on success, a 0 otherwise.
\end{description}\\

\item[\textbf{pyp\_file\_footer(ofhandle, caller="Unknown PyPedal routine")} [\#]
]
\par pyp\_file\_footer() writes a footer to a page of PyPedal output.
\begin{description}
\item[\textit{ofhandle}
]
A Python file handle.
\item[\textit{caller}
]
A string indicating the name of the calling routine.
\item[Returns:
]
None
\end{description}\\

\item[\textbf{pyp\_file\_header(ofhandle, caller="Unknown PyPedal routine")} [\#]
]
\par pyp\_file\_header() writes a header to a page of PyPedal output.
\begin{description}
\item[\textit{ofhandle}
]
A Python file handle.
\item[\textit{caller}
]
A string indicating the name of the calling routine.
\item[Returns:
]
None
\end{description}\\

\item[\textbf{renderBodyText(text\_string)} [\#]
]
\par renderBodyText() renders page contents (produces HTML output by default).

\item[\textbf{renderTitle(title\_string, title\_level="1")} [\#]
]
\par renderTitle() renders page titles (produces HTML output by default).

\item[\textbf{save\_from\_gedcom(outfilename, assembled)} ⇒ string [\#]
]
\par save\_from\_gedcom() takes pedigree data parsed by load\_from\_gedcom() and
writes it to a text file in an ASD format that PyPedal can easily read.
\begin{description}
\item[\textit{outfilename}
]
The file to which the matrix should be written.
\item[\textit{fam2ids}
]
Dictionary mapping family IDs to parents and offspring
\item[\textit{missing\_parent}
]
Value assigned to unknown parents
\item[\textit{id2name}
]
Dictionary mapping IDs to names
\item[\textit{id2birth}
]
Dictionary mapping IDs to birth years
\item[Returns:
]
A string containing the pedigree format code. 'xxxx' if there was a problem.
\end{description}\\

\item[\textbf{save\_ijk(pedobj, nrm\_filename)} ⇒ integer [\#]
]
\par save\_ijk() saves an NRM to a file in the form "animal A" "animal B" "rAB".
\begin{description}
\item[\textit{nrm\_filename}
]
The file to which the matrix should be written.
\item[Returns:
]
A save status indicator (0: failed, 1: success).
\end{description}\\

\item[\textbf{save\_to\_gedcom(pedobj, outfilename)} ⇒ integer [\#]
]
\par save\_to\_gedcom() writes a PyPedal NewPedigree object to a file in
GEDCOM 5.5 format.
\begin{description}
\item[\textit{pedobj}
]
An instance of a PyPedal NewPedigree object
\item[\textit{filename}
]
The file to which the matrix should be written
\item[Returns:
]
A save status indicator (0: failed, 1: success).
\end{description}\\

\item[\textbf{summary\_inbreeding(f\_metadata)} ⇒ string [\#]
]
\par summary\_inbreeding() returns a string representation of the data contained in
the 'metadata' dictionary contained in the output dictionary returned by
pyp\_nrm/pyp\_inbreeding().
\begin{description}
\item[\textit{f\_metadata}
]
Dictionary of inbreeding metadata.
\item[Returns:
]
A string on success, a 0 otherwise.
\end{description}\\

\item[\textbf{unpickle\_pedigree(filename='')} ⇒ object [\#]
]
\par unpickle\_pedigree() reads a pickled pedigree in from a file and returns the unpacked
pedigree object.
\begin{description}
\item[\textit{filename}
]
The name of the pickle file.
\item[Returns:
]
An instance of a NewPedigree object on success, a 0 otherwise.
\end{description}\\

\end{description}
\end{document}
