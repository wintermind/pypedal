\par pyp\_reports contains a set of procedures for ...
\subsection*{Module Contents}
\begin{description}
\item[\textbf{\_pdfCreateTitlePage(canv, \_pdfSettings, reporttitle='', reportauthor='')} ⇒ None [\#]
]
\par \_pdfCreateTitlePage() adds a title page to a ReportLab canvas object.
\begin{description}
\item[\textit{canv}
]
An instance of a ReportLab Canvas object.
\item[\textit{\_pdfSettings}
]
An options dictionary created by \_pdfInitialize().
\item[Returns:
]
None
\end{description}\\

\item[\textbf{\_pdfDrawPageFrame(canv, \_pdfSettings)} ⇒ None [\#]
]
\par \_pdfDrawPageFrame() nicely frames page contents and includes the
document title in a header and the page number in a footer.
\begin{description}
\item[\textit{canv}
]
An instance of a ReportLab Canvas object.
\item[\textit{\_pdfSettings}
]
An options dictionary created by \_pdfInitialize().
\item[Returns:
]
None
\end{description}\\

\item[\textbf{\_pdfInitialize(pedobj)} ⇒ dictionary [\#]
]
\par \_pdfInitialize() returns a dictionary of metadata that is used for report
generation.
\begin{description}
\item[\textit{pedobj}
]
A PyPedal pedigree object.
\item[Returns:
]
A dictionary of metadata that is used for report generation.
\end{description}\\

\item[\textbf{meanMetricBy(pedobj, metric='fa', byvar='by', createpdf=0)} ⇒ dictionary [\#]
]
\par meanMetricBy() returns a dictionary of means keyed by levels of the 'byvar' that
can be used to draw graphs or prepare reports of summary statistics.
\begin{description}
\item[\textit{pedobj}
]
A PyPedal pedigree object.
\item[\textit{metric}
]
The variable to summarize on a BY variable.
\item[\textit{byvar}
]
The variable on which to group the metric.
\item[\textit{createpdf}
]
Flag indicating whether or not a PDF version of the report should be created.
\item[Returns:
]
A dictionary containing means for the metric variable keyed to levels of the byvar.
\end{description}\\

\item[\textbf{pdf3GenPed(animalID, pedobj, titlepage=0, reporttitle='', reportauthor='', reportfile='')} ⇒ integer [\#]
]
\par pdf3GenPed() draws a three-generation pedigree for animal 'animalID'.
\begin{description}
\item[\textit{animalID}
]
An animal ID or list of animal IDs.
\item[\textit{pedobj}
]
A PyPedal pedigree object.
\item[\textit{titlepage}
]
Show (1) or hide (0) the title page.
\item[\textit{reporttitle}
]
Title of report; if '', \_pdfTitle is used.
\item[\textit{reportauthor}
]
Author/preparer of report.
\item[\textit{reportfile}
]
Optional name of file to which the report should be written.
\item[Returns:
]
1 on success, 0 on failure
\end{description}\\

\item[\textbf{pdfMeanMetricBy(pedobj, results, titlepage=0, reporttitle='', reportauthor='', reportfile='')} ⇒ integer [\#]
]
\par pdfMeanMetricBy() returns a dictionary of means keyed by levels of the 'byvar' that
can be used to draw graphs or prepare reports of summary statistics.
\begin{description}
\item[\textit{pedobj}
]
A PyPedal pedigree object.
\item[\textit{results}
]
A dictionary containing means for the metric variable keyed to levels of the byvar.
\item[\textit{titlepage}
]
Show (1) or hide (0) the title page.
\item[\textit{reporttitle}
]
Title of report; if '', \_pdfTitle is used.
\item[\textit{reportauthor}
]
Author/preparer of report.
\item[\textit{reportfile}
]
Optional name of file to which the report should be written.
\item[Returns:
]
1 on success, 0 on failure
\end{description}\\

\item[\textbf{pdfPedigreeMetadata(pedobj, titlepage=0, reporttitle='', reportauthor='', reportfile='')} ⇒ integer [\#]
]
\par pdfPedigreeMetadata() produces a report, in PDF format, of the metadata from
the input pedigree.  It is intended for use as a template for custom printed
reports.
\begin{description}
\item[\textit{pedobj}
]
A PyPedal pedigree object.
\item[\textit{titlepage}
]
Show (1) or hide (0) the title page.
\item[\textit{reporttitle}
]
Title of report; if '', \_pdfTitle is used.
\item[\textit{reportauthor}
]
Author/preparer of report.
\item[\textit{reportfile}
]
Optional name of file to which the report should be written.
\item[Returns:
]
A 1 on success, 0 otherwise.
\end{description}\\

\end{description}
