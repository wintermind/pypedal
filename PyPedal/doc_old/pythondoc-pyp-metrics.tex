\par pyp\_metrics contains a set of procedures for calculating metrics on PyPedal
pedigree objects.  These metrics include coefficients of inbreeding and
relationship as well as effective founder number, effective population size,
and effective ancestor number.
\begin{description}
\item[\textbf{a\_coefficients(pedobj, a='', method='nrm')} ⇒ dictionary [\#]
]
\par a\_coefficients() writes population average coefficients of inbreeding and
relationship to a file, as well as individual animal IDs and coefficients of
inbreeding.  Some pedigrees are too large for fast\_a\_matrix() or fast\_a\_matrix\_r()
-$\,$- an array that large cannot be allocated due to memory restrictions -$\,$- and will
result in a value of -999.9 for all outputs.
\begin{description}
\item[\textit{pedobj}
]
A PyPedal pedigree object.
\item[\textit{a}
]
A numerator relationship matrix (optional).
\item[\textit{method}
]
If no relationship matrix is passed, determines which procedure should be called to build one (nrm|frm).
\item[Returns:
]
A dictionary of non-zero individual inbreeding coefficients.
\end{description}\\

\item[\textbf{a\_effective\_ancestors\_definite(pedobj, a='', gen='')} ⇒ float [\#]
]
\par a\_effective\_ancestors\_definite() uses the algorithm in Appendix B of Boichard et al.
(1996) to compute the effective ancestor number for a myped pedigree.
NOTE: One problem here is that if you pass a pedigree WITHOUT generations and error
is not thrown.  You simply end up wth a list of generations that contains the default
value for Animal() objects, 0.
Boichard's algorithm requires information about the GENERATION of animals.  If you
do not provide an input pedigree with generations things may not work.  By default
the most recent generation -$\,$- the generation with the largest generation ID -$\,$- will
be used as the reference population.
\begin{description}
\item[\textit{pedobj}
]
A PyPedal pedigree object.
\item[\textit{a}
]
A numerator relationship matrix (optional).
\item[\textit{gen}
]
Generation of interest.
\item[Returns:
]
The effective ancestor number.
\end{description}\\

\item[\textbf{a\_effective\_ancestors\_indefinite(pedobj, a='', gen='', n=25)} ⇒ float [\#]
]
\par a\_effective\_ancestors\_indefinite() uses the approach outlined on pages 9 and 10 of
Boichard et al. (1996) to compute approximate upper and lower bounds for f\_a.  This
is much more tractable for large pedigrees than the exact computation provided in
a\_effective\_ancestors\_definite().
NOTE: One problem here is that if you pass a pedigree WITHOUT generations and error
is not thrown.  You simply end up wth a list of generations that contains the default
value for Animal() objects, 0.
NOTE: If you pass a value of n that is greater than the actual number of ancestors in
the pedigree then strange things happen.  As a stop-gap, a\_effective\_ancestors\_indefinite()
will detect that case and replace n with the number of founders - 1.
Boichard's algorithm requires information about the GENERATION of animals.  If you
do not provide an input pedigree with generations things may not work.  By default
the most recent generation -$\,$- the generation with the largest generation ID -$\,$- will
be used as the reference population.
\begin{description}
\item[\textit{pedobj}
]
A PyPedal pedigree object.
\item[\textit{a}
]
A numerator relationship matrix (optional).
\item[\textit{gen}
]
Generation of interest.
\item[Returns:
]
The effective ancestor number.
\end{description}\\

\item[\textbf{a\_effective\_founders\_boichard(pedobj, a='', gen='')} ⇒ float [\#]
]
\par a\_effective\_founders\_boichard() uses the algorithm in Appendix A of Boichard et al.
(1996) to compute the effective founder number for myped.  Note that results from
this function will not necessarily match those from a\_effective\_founders\_lacy().
Boichard's algorithm requires information about the GENERATION of animals.  If you
do not provide an input pedigree with generations things may not work.  By default
the most recent generation -$\,$- the generation with the largest generation ID -$\,$- will
be used as the reference population.
\begin{description}
\item[\textit{pedobj}
]
A PyPedal pedigree object.
\item[\textit{a}
]
A numerator relationship matrix (optional).
\item[\textit{gen}
]
Generation of interest.
\item[Returns:
]
The effective founder number.
\end{description}\\

\item[\textbf{a\_effective\_founders\_lacy(pedobj, a='')} ⇒ dictionary [\#]
]
\par a\_effective\_founders\_lacy() calculates the number of effective founders in a
pedigree using the exact method of Lacy.
\begin{description}
\item[\textit{pedobj}
]
A PyPedal pedigree object.
\item[\textit{a}
]
A numerator relationship matrix (optional).
\item[Returns:
]
A dictionary of results, including the effective founder number.
\end{description}\\

\item[\textbf{ballou\_ancestral\_inbreeding(pedobj)} ⇒ dictionary [\#]
]
\par ballou\_ancestral\_inbreeding() calculates ancestral inbreeding,
the probability of an individual inheriting an allele that has
undergone inbreeding in the past at least once, using the method
of Ballou (1997).
\begin{description}
\item[\textit{pedobj}
]
A PyPedal pedigree object.
\item[Returns:
]
A dictionary of ancestral inbreeding coefficients keyed to animal IDs.
\end{description}\\

\item[\textbf{common\_ancestors(anim\_a, anim\_b, pedobj)} ⇒ list [\#]
]
\par common\_ancestors() returns a list of the ancestors that two animals share in common.
\begin{description}
\item[\textit{anim\_a}
]
The renumbered ID of the first animal, a.
\item[\textit{anim\_b}
]
The renumbered ID of the second animal, b.
\item[\textit{pedobj}
]
A PyPedal pedigree object.
\item[Returns:
]
A list of animals related to anim\_a AND anim\_b
\end{description}\\

\item[\textbf{descendants(anid, pedobj, \_desc)} ⇒ list [\#]
]
\par descendants() uses pedigree metadata to walk a pedigree and return a list of all
of the descendants of a given animal.
\begin{description}
\item[\textit{anid}
]
An animal ID
\item[\textit{pedobj}
]
A Python list of  PyPedal Animal() objects.
\item[\textit{\_desc}
]
A Python dictionary of descendants of animal anid.
\item[Returns:
]
A list of descendants of anid.
\end{description}\\

\item[\textbf{dropped\_ancestral\_inbreeding(pedobj, rounds=100, loci=100, frequency=0.05, seed=5048665)} ⇒ dictionary [\#]
]
\par dropped\_ancestral\_inbreeding() uses a gene dropping approach to calculate
ancestral inbreeding, the probability of an individual inheriting an allele
that has undergone inbreeding in the past at least once.
\begin{description}
\item[\textit{pedobj}
]
A PyPedal pedigree object.
\item[\textit{rounds}
]
The number of times to simulate segregation through the entire pedigree.
\item[\textit{loci}
]
The number biallelic, unlinked loci to simulate.
\item[\textit{frequency}
]
The minor allele frequency.
\item[\textit{seed}
]
The seed for the RNG.
\item[Returns:
]
A dictionary of ancestral inbreeding coefficients keyed to animal IDs.
\end{description}\\

\item[\textbf{effective\_founder\_genomes(pedobj, rounds=10)} ⇒ float [\#]
]
\par effective\_founder\_genomes() simulates the random segregation of founder alleles through a pedigree.
At present only two alleles are simulated for each founder.  Summary statistics are
computed on the most recent generation.
\begin{description}
\item[\textit{pedobj}
]
A PyPedal pedigree object.
\item[\textit{rounds}
]
The number of times to simulate segregation through the entire pedigree.
\item[Returns:
]
The effective number of founder genomes over based on 'rounds' gene-drop simulations.
\end{description}\\

\item[\textbf{effective\_founders\_lacy(pedobj)} ⇒ dictionary [\#]
]
\par effective\_founders\_lacy() calculates the number of effective founders in a pedigree
using the exact method of Lacy.  This version of the routine a\_effective\_founders\_lacy()
is designed to work with larger pedigrees as it forms "familywise" relationship matrices
rather than a "populationwise" relationship matrix.
\begin{description}
\item[\textit{pedobj}
]
A PyPedal pedigree object.
\item[Returns:
]
A dictionary of results, including the effective founder number.
\end{description}\\

\item[\textbf{fast\_a\_coefficients(pedobj, a='', method='nrm', debug=0, storage='dense')} ⇒ dictionary [\#]
]
\par a\_fast\_coefficients() writes population average coefficients of inbreeding and
relationship to a file, as well as individual animal IDs and coefficients of
inbreeding.  It returns a list of non-zero individual CoI.
\begin{description}
\item[\textit{pedobj}
]
A PyPedal pedigree object.
\item[\textit{a}
]
A numerator relationship matrix (optional).
\item[\textit{method}
]
If no relationship matrix is passed, determines which procedure should be called to build one (nrm|frm).
\item[\textit{storage}
]
Use dense or sparse matrix storage.
\item[Returns:
]
A dictionary of non-zero individual inbreeding coefficients.
\end{description}\\

\item[\textbf{founder\_descendants(pedobj)} ⇒ dictionary [\#]
]
\par founder\_descendants() returns a dictionary containing a list of descendants of
each founder in the pedigree.
\begin{description}
\item[\textit{pedojb}
]
An instance of a PyPedal NewPedigree object.
\end{description}\\

\item[\textbf{generation\_intervals(pedobj, units='y')} ⇒ dictionary [\#]
]
\par generation\_intervals() computes the average age of parents at the time of
birth of their first (oldest) offspring.  This is implies that selection
decisions are made at the time of birth of the first offspring.  Average
ages are computed for each of four paths: sire-son, sire-daughter, dam-son,
and dam-daughter.  An overall mean is computed, as well.  IT IS IMPORTANT
to note that if you DO NOT provide birthyears in your pedigree file that the
returned dictionary will contain only zeroes!  This is because when no birthyear
is provided a default value (1900) is assigned to all animals in the pedigree.
\begin{description}
\item[\textit{pedobj}
]
A PyPedal pedigree object.
\item[\textit{units}
]
A character indicating the units in which the generation lengths should be returned.
\item[Returns:
]
A dictionary containing the five average ages.
\end{description}\\

\item[\textbf{generation\_intervals\_all(pedobj, units='y')} ⇒ dictionary [\#]
]
\par generation\_intervals\_all() computes the average age of parents at the time of
birth of their offspring.  The computation is made using birth years for all
known offspring of sires and dams, which implies discrete generations.  Average
ages are computed for each of four paths: sire-son, sire-daughter, dam-son, and
dam-daughter.  An overall mean is computed, as well. IT IS IMPORTANT to note that
if you DO NOT provide birthyears in your pedigree file that the returned dictionary
will contain only zeroes!  This is because when no birthyear is provided a default
value (1900) is assigned to all animals in the pedigree.
\begin{description}
\item[\textit{pedobj}
]
A PyPedal pedigree object.
\item[\textit{units}
]
A character indicating the units in which the generation lengths should be returned.
\item[Returns:
]
A dictionary containing the five average ages.
\end{description}\\

\item[\textbf{mating\_coi(anim\_a, anim\_b, pedobj, gens=0)} ⇒ float [\#]
]
\par mating\_coi() returns the coefficient of inbreeding of offspring of a
mating between two animals, anim\_a and anim\_b.
\begin{description}
\item[\textit{anim\_a}
]
The renumbered ID of an animal, a.
\item[\textit{anim\_b}
]
The renumbered ID of an animal, b.
\item[\textit{pedobj}
]
A PyPedal pedigree object.
\item[\textit{gens}
]
The number of generations from the pedigree to be used for calculating CoI.  By default, gens=-1, which uses the complete pedigree.
\item[Returns:
]
The coefficient of inbreeding of the offpsring of anim\_a and anim\_b
\end{description}\\

\item[\textbf{mating\_coi\_group(matings, pedobj, names=0, gens=0)} ⇒ dictionary [\#]
]
\par mating\_coi\_group() returns the coefficients of inbreeding of offspring
of a series of matings, as well as a list of minimum-inbreeding matings.
\begin{description}
\item[\textit{matings}
]
A dictionary
\item[\textit{pedobj}
]
A PyPedal pedigree object.
\item[\textit{names}
]
Indicates if the identifiers in 'matings' are names or animalIDs
\item[\textit{gens}
]
The number of generations from the pedigree to be used for calculating CoI.  By default, gens=-1, which uses the complete pedigree.
\item[Returns:
]
he coefficient of inbreeding of the offpsring of anim\_a and anim\_b
\end{description}\\

\item[\textbf{min\_max\_f(pedobj, a='', n=10, forma='dense')} ⇒ list [\#]
]
\par min\_max\_f() takes a pedigree and returns a list of the individuals with the n
largest and n smallest coefficients of inbreeding.  Individuals with CoI of
zero are not included.
\begin{description}
\item[\textit{pedobj}
]
A PyPedal pedigree object.
\item[\textit{a}
]
A numerator relationship matrix (optional).
\item[\textit{n}
]
An integer (optional, default is 10).
\item[\textit{forma}
]
If A must be formed should dense or sparse matrices be used?
\item[Returns:
]
Lists of the individuals with the n largest and the  n smallest CoI in the pedigree as (ID, CoI) tuples.
\end{description}\\

\item[\textbf{pedigree\_completeness(pedobj, gens=4)} ⇒ dictionary [\#]
]
\par pedigree\_completeness() computes the proportion of known ancestors in the pedigree of
each animal in the population for a user-determined number of generations.    Also,
the mean pedcomps for all animals and for all animals that are not founders are
computed as summary statistics.
\begin{description}
\item[\textit{pedobj}
]
A PyPedal pedigree object.
\item[\textit{gens}
]
The number of generations the pedigree should be traced for completeness.
\item[Returns:
]
Dictionary of summary statistics
\end{description}\\

\item[\textbf{related\_animals(anim, pedobj)} ⇒ list [\#]
]
\par related\_animals() returns a list of the ancestors of an animal.
\begin{description}
\item[\textit{anim\_a}
]
The renumbered ID of an animal, a.
\item[\textit{pedobj}
]
A PyPedal pedigree object.
\item[Returns:
]
A list of animals related to anim\_a
\end{description}\\

\item[\textbf{relationship(anim\_a, anim\_b, pedobj)} ⇒ float [\#]
]
\par relationship() returns the coefficient of relationship for two
animals, anim\_a and anim\_b.
\begin{description}
\item[\textit{anim\_a}
]
The renumbered ID of an animal, a.
\item[\textit{anim\_b}
]
The renumbered ID of an animal, b.
\item[\textit{pedobj}
]
A PyPedal pedigree object.
\item[Returns:
]
The coefficient of relationship of anim\_a and anim\_b
\end{description}\\

\item[\textbf{theoretical\_ne\_from\_metadata(pedobj)} ⇒ integer [\#]
]
\par theoretical\_ne\_from\_metadata() computes the theoretical effective population
size based on the number of sires and dams contained in a pedigree metadata
object.  Writes results to an output file.
\begin{description}
\item[\textit{pedobj}
]
A PyPedal pedigree object.
\item[Returns:
]
True (1) on success, false (0) on failure
\end{description}\\

\end{description}
