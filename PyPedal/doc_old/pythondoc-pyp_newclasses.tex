% This file was converted from HTML to LaTeX with
% Tomasz Wegrzanowski's <maniek@beer.com> gnuhtml2latex program
% Version : 0.1
\documentclass{article}
\begin{document}

\section*{The pyp\_newclasses Module}
\par pyp\_newclasses contains the new class structure that will be a part of PyPedal 2.0.0Final.
It includes a master class to which most of the computational routines will be bound as
methods, a NewAnimal() class, and a PedigreeMetadata() class.
\subsection*{Module Contents}
\begin{description}
\item[\textbf{LightAnimal(locations, data, mykw)} (class) [\#]
]
\par The LightAnimal() class holds animals records read from a pedigree file.
\par For more information about this class, see \textit{The LightAnimal Class}.

\item[\textbf{loadPedigree(options='', optionsfile='pypedal.ini', pedsource='file', pedgraph=0)} ⇒ PyPedal pedigree object [\#]
]
\par loadPedigree() wraps pedigree creation and loading into a one-step
process.  If the user passes both a dictionary and a filename, the
dictionary will be used instead of the filename unless the dictionary
is empty.
\begin{description}
\item[\textit{options}
]
Dictionary of pedigree options.
\item[\textit{optionsfile}
]
File from which pedigree options should be read.
\item[\textit{pedsource}
]
Source of the pedigree ('file'|'graph'|'graphfile'|'db').
\item[\textit{pedgraph}
]
XDiGraph from which to load the pedigree.
\item[Returns:
]
An instance of a NewPedigree object on success, a 0 on failure.
\end{description}\\

\item[\textbf{NewAMatrix(kw)} (class) [\#]
]
\par NewAMatrix provides an instance of a numerator relationship matrix as a Numarray array of
floats with some convenience methods.
\par For more information about this class, see \textit{The NewAMatrix Class}.

\item[\textbf{NewAnimal(locations, data, mykw)} (class) [\#]
]
\par The NewAnimal() class is holds animals records read from a pedigree file.
\par For more information about this class, see \textit{The NewAnimal Class}.

\item[\textbf{NewPedigree(kw=\{\}, kwfile='pypedal.ini')} (class) [\#]
]
\par The NewPedigree class is the main data structure for PyP 2.0.0Final.
\par For more information about this class, see \textit{The NewPedigree Class}.

\item[\textbf{PedigreeMetadata(myped, kw)} (class) [\#]
]
\par The PedigreeMetadata() class stores metadata about pedigrees.
\par For more information about this class, see \textit{The PedigreeMetadata Class}.

\item[\textbf{PyPedalError} (class)  [\#]
]
\par PyPedalError is the base class for exceptions in PyPedal.
\par For more information about this class, see \textit{The PyPedalError Class}.

\item[\textbf{PyPedalPedigreeInputFileNameError()} (class) [\#]
]
\par PyPedalPedigreeInputFileNameError is raised when a simulated pedigree
is not requested and a pedigree file name is not provided.
\par For more information about this class, see \textit{The PyPedalPedigreeInputFileNameError Class}.

\item[\textbf{SimAnimal(animalID, sireID=0, damID=0, sex='u', gen=0)} (class) [\#]
]
\par The SimAnimal() class is a placeholder used for simulating animals.
\par For more information about this class, see \textit{The SimAnimal Class}.

\item[\textbf{tail\_recursive(func)} (class) [\#]
]
\par tail\_recursive is a tail recursion decorator that eliminates tail calls
for recursive functions.
\par For more information about this class, see \textit{The tail\_recursive Class}.

\end{description}
\subsection*{The LightAnimal Class}
\begin{description}
\item[\textbf{LightAnimal(locations, data, mykw)} (class) [\#]
]
\par The LightAnimal() class holds animals records read from a pedigree file. It
is a much simpler object than the NewAnimal() object and is intended for use
with the graph theoretic routines in pyp\_network. The only attributes of these
objects are: animal ID, sire ID, dam ID, original ID, birth year, and sex.

\item[\textbf{\_\_init\_\_(locations, data, mykw)} ⇒ object [\#]
]
\par \_\_init\_\_() initializes a LightAnimal() object.
\begin{description}
\item[\textit{locations}
]
A dictionary containing the locations of variables in the input line.
\item[\textit{data}
]
The line of input read from the pedigree file.
\item[Returns:
]
An instance of a LightAnimal() object populated with data
\end{description}\\

\item[\textbf{dictme()} [\#]
]
\par dictme() returns a summary of the data stored in the NewAnimal() object
as a dictionary.
\begin{description}
\item[\textit{self}
]
Reference to the current NewAnimal() object
\end{description}\\

\item[\textbf{pad\_id()} ⇒ integer [\#]
]
\par pad\_id() takes an Animal ID, pads it to fifteen digits, and prepends the birthyear
(or 1950 if the birth year is unknown).  The order of elements is: birthyear, animalID,
count of zeros, zeros.
\begin{description}
\item[\textit{self}
]
Reference to the current LightAnimal() object
\item[Returns:
]
A padded ID number that is supposed to be unique across animals
\end{description}\\

\item[\textbf{printme()} [\#]
]
\par printme() prints a summary of the data stored in the LightAnimal() object.
\begin{description}
\item[\textit{self}
]
Reference to the current LightAnimal() object
\end{description}\\

\item[\textbf{string\_to\_int(idstring, mymaxint=9223372036854775807)} [\#]
]
\par string\_to\_int() takes an Animal/Sire/Dam ID as a string and returns a
hash.

\item[\textbf{stringme()} [\#]
]
\par stringme() returns a summary of the data stored in the LightAnimal() object
as a string.
\begin{description}
\item[\textit{self}
]
Reference to the current LightAnimal() object
\end{description}\\

\item[\textbf{trap()} [\#]
]
\par trap() checks for common errors in LightAnimal() objects
\begin{description}
\item[\textit{self}
]
Reference to the current LightAnimal() object
\end{description}\\

\end{description}
\subsection*{The NewAMatrix Class}
\begin{description}
\item[\textbf{NewAMatrix(kw)} (class) [\#]
]
\par NewAMatrix provides an instance of a numerator relationship matrix as a Numarray array of
floats with some convenience methods.  The idea here is to provide a wrapper around a NRM
so that it is easier to work with.  For large pedigrees it can take a long time to compute
the elements of A, so there is real value in providing an easy way to save and retrieve a
NRM once it has been formed.

\item[\textbf{\_\_init\_\_(kw)} ⇒ object [\#]
]
\par \_\_init\_\_() initializes a NewAMatrix object.
\begin{description}
\item[\textit{self}
]
Reference to the current NewAMatrix() object
\item[\textit{kw}
]
A dictionary of options.
\item[Returns:
]
An instance of a NewAMatrix() object
\end{description}\\

\item[\textbf{form\_a\_matrix(pedigree)} ⇒ integer [\#]
]
\par form\_a\_matrix() calls pyp\_nrm/fast\_a\_matrix() or pyp\_nrm/fast\_a\_matrix\_r()
to form a NRM from a pedigree.
\begin{description}
\item[\textit{pedigree}
]
The pedigree used to form the NRM.
\item[Returns:
]
A NRM on success, 0 on failure.
\end{description}\\

\item[\textbf{info()} ⇒ None [\#]
]
\par info() uses the info() method of Numarray arrays to dump some information about
the NRM.  This is of use predominantly for debugging.
\begin{description}
\item[\textit{None}
]

\item[Returns:
]
None
\end{description}\\

\item[\textbf{load(nrm\_filename)} ⇒ integer [\#]
]
\par load() uses the Numarray Array Function "fromfile()" to load an array from a
binary file.  If the load is successful, self.nrm contains the matrix.
\begin{description}
\item[\textit{nrm\_filename}
]
The file from which the matrix should be read.
\item[Returns:
]
A load status indicator (0: failed, 1: success).
\end{description}\\

\item[\textbf{printme()} ⇒ None [\#]
]
\par printme() prints the NRM to the screen.
\begin{description}
\item[\textit{None}
]

\item[Returns:
]
None
\end{description}\\

\item[\textbf{save(nrm\_filename, nrm\_format='')} ⇒ integer [\#]
]
\par save() uses the Numarray method "tofile()" to save an array to a binary file.
\begin{description}
\item[\textit{nrm\_filename}
]
The file to which the matrix should be written.
\item[Returns:
]
A save status indicator (0: failed, 1: success).
\end{description}\\

\end{description}
\subsection*{The NewAnimal Class}
\begin{description}
\item[\textbf{NewAnimal(locations, data, mykw)} (class) [\#]
]
\par The NewAnimal() class is holds animals records read from a pedigree file.

\item[\textbf{\_\_init\_\_(locations, data, mykw)} ⇒ object [\#]
]
\par \_\_init\_\_() initializes a NewAnimal() object.
\begin{description}
\item[\textit{locations}
]
A dictionary containing the locations of variables in the input line.
\item[\textit{data}
]
The line of input read from the pedigree file.
\item[Returns:
]
An instance of a NewAnimal() object populated with data
\end{description}\\

\item[\textbf{dictme()} [\#]
]
\par dictme() returns a summary of the data stored in the NewAnimal() object
as a dictionary.
\begin{description}
\item[\textit{self}
]
Reference to the current NewAnimal() object
\end{description}\\

\item[\textbf{pad\_id()} ⇒ integer [\#]
]
\par pad\_id() takes an Animal ID, pads it to fifteen digits, and prepends the birthyear
(or 1950 if the birth year is unknown).  The order of elements is: birthyear, animalID,
count of zeros, zeros.
\begin{description}
\item[\textit{self}
]
Reference to the current NewAnimal() object
\item[Returns:
]
A padded ID number that is supposed to be unique across animals
\end{description}\\

\item[\textbf{printme()} [\#]
]
\par printme() prints a summary of the data stored in the NewAnimal() object.
\begin{description}
\item[\textit{self}
]
Reference to the current NewAnimal() object
\end{description}\\

\item[\textbf{string\_to\_int(idstring, mymaxint=9223372036854775807)} [\#]
]
\par string\_to\_int() takes an Animal/Sire/Dam ID as a string and returns a
hash.

\item[\textbf{stringme()} [\#]
]
\par stringme() returns a summary of the data stored in the NewAnimal() object
as a string.
\begin{description}
\item[\textit{self}
]
Reference to the current NewAnimal() object
\end{description}\\

\item[\textbf{trap()} [\#]
]
\par trap() checks for common errors in NewAnimal() objects
\begin{description}
\item[\textit{self}
]
Reference to the current NewAnimal() object
\end{description}\\

\end{description}
\subsection*{The NewPedigree Class}
\begin{description}
\item[\textbf{NewPedigree(kw=\{\}, kwfile='pypedal.ini')} (class) [\#]
]
\par The NewPedigree class is the main data structure for PyP 2.0.0Final.

\item[\textbf{\_\_init\_\_(kw=\{\}, kwfile='pypedal.ini')} ⇒ object [\#]
]
\par \_\_init\_\_() initializes a NewPedigree object.
\begin{description}
\item[\textit{self}
]
Reference to the current NewPedigree() object
\item[\textit{kw}
]
A dictionary of options.
\item[\textit{kwfile}
]
An optionsl configuration file name
\item[Returns:
]
An instance of a NewPedigree() object
\end{description}\\

\item[\textbf{addanimal(animalID, sireID, damID)} ⇒ integer [\#]
]
\par addanimal() adds a new animal of class NewAnimal to the pedigree.
\begin{description}
\item[\textit{}
]

\item[Returns:
]
1 on success, 0 on failure
\end{description}\\

\item[\textbf{delanimal(animalID)} ⇒ integer [\#]
]
\par delanimal() deletes an animal from the pedigree. Note that this
method DOES not update the metadata attached to the pedigree
and should only be used if that is not important. As of 04/10/2006
delanimal() is intended for use by pyp\_metrics/mating\_coi() rather
than directly by users.
\begin{description}
\item[\textit{animalID}
]
ID of the animal to be deleted
\item[Returns:
]
1 on success, 0 on failure
\end{description}\\

\item[\textbf{fromgraph(pedgraph)} ⇒ None [\#]
]
\par fromgraph() loads the animals to populate the pedigree from an
XDiGraph object.
\begin{description}
\item[\textit{pedgraph}
]

\item[Returns:
]
None
\end{description}\\

\item[\textbf{load(pedsource='file', pedgraph=0)} ⇒ None [\#]
]
\par load() wraps several processes useful for loading and preparing a pedigree for
use in an analysis, including reading the animals into a list of animal objects,
forming lists of sires and dams, checking for common errors, setting ancestor
flags, and renumbering the pedigree.
\begin{description}
\item[\textit{pedsource}
]
Source of the pedigree ('file'|'graph'|'graphfile'|'db').
\item[\textit{pedgraph}
]
XDiGraph from which to load the pedigree.
\item[Returns:
]
None
\end{description}\\

\item[\textbf{preprocess()} ⇒ None [\#]
]
\par preprocess() processes a pedigree file, which includes reading the animals
into a list of animal objects, forming lists of sires and dams, and checking for
common errors.
\begin{description}
\item[\textit{None}
]

\item[Returns:
]
None
\end{description}\\

\item[\textbf{printoptions()} ⇒ None [\#]
]
\par printoptions() prints the contents of the options dictionary.
\begin{description}
\item[\textit{None}
]

\item[Returns:
]
None
\end{description}\\

\item[\textbf{renumber()} ⇒ None [\#]
]
\par renumber() updates the ID map after a pedigree has been renumbered so that all references
are to renumbered rather than original IDs.
\begin{description}
\item[\textit{None}
]

\item[Returns:
]
None
\end{description}\\

\item[\textbf{save(filename='', outformat='o', idformat='o')} ⇒ integer [\#]
]
\par save() writes a PyPedal pedigree to a user-specified file.  The saved pedigree includes
all fields recognized by PyPedal, not just the original fields read from the input pedigree
file.
\begin{description}
\item[\textit{filename}
]
The file to which the pedigree should be written.
\item[\textit{outformat}
]
The format in which the pedigree should be written: 'o' for original (as read) and 'l' for long version (all available variables).
\item[\textit{idformat}
]
Write 'o' (original) or 'r' (renumbered) animal, sire, and dam IDs.
\item[Returns:
]
A save status indicator (0: failed, 1: success)
\end{description}\\

\item[\textbf{simulate()} ⇒ None [\#]
]
\par simulate() simulates an arbitrary pedigree of size n with g generations
starting from n\_s base sires and n\_d base dams.  This method is based on
the concepts and algorithms in the Pedigree::sample method from Matvec
1.1a.  The arguments are read from the pedigree object's options
dictionary.
\begin{description}
\item[\textit{None}
]

\item[Returns:
]
None
\end{description}\\

\item[\textbf{updateidmap()} ⇒ None [\#]
]
\par updateidmap() updates the ID map after a pedigree has been renumbered so that all references
are to renumbered rather than original IDs.
\begin{description}
\item[\textit{None}
]

\item[Returns:
]
None
\end{description}\\

\end{description}
\subsection*{The PedigreeMetadata Class}
\begin{description}
\item[\textbf{PedigreeMetadata(myped, kw)} (class) [\#]
]
\par The PedigreeMetadata() class stores metadata about pedigrees.  Hopefully this will help improve performance in some procedures,
as well as provide some useful summary data.

\item[\textbf{\_\_init\_\_(myped, kw)} ⇒ object [\#]
]
\par \_\_init\_\_() initializes a PedigreeMetadata object.
\begin{description}
\item[\textit{self}
]
Reference to the current Pedigree() object
\item[\textit{myped}
]
A PyPedal pedigree.
\item[\textit{kw}
]
A dictionary of options.
\item[Returns:
]
An instance of a Pedigree() object populated with data
\end{description}\\

\item[\textbf{fileme()} [\#]
]
\par fileme() writes the metada stored in the Pedigree() object to disc.
\begin{description}
\item[\textit{self}
]
Reference to the current Pedigree() object
\end{description}\\

\item[\textbf{nud()} ⇒ integer-and-list [\#]
]
\par nud() returns the number of unique dams in the pedigree along with a list of the dams
\begin{description}
\item[\textit{self}
]
Reference to the current Pedigree() object
\item[Returns:
]
The number of unique dams in the pedigree and a list of those dams
\end{description}\\

\item[\textbf{nuf()} ⇒ integer-and-list [\#]
]
\par nuf() returns the number of unique founders in the pedigree along with a list of the founders
\begin{description}
\item[\textit{self}
]
Reference to the current Pedigree() object
\item[Returns:
]
The number of unique founders in the pedigree and a list of those founders
\end{description}\\

\item[\textbf{nug()} ⇒ integer-and-list [\#]
]
\par nug() returns the number of unique generations in the pedigree along with a list of the generations
\begin{description}
\item[\textit{self}
]
Reference to the current Pedigree() object
\item[Returns:
]
The number of unique generations in the pedigree and a list of those generations
\end{description}\\

\item[\textbf{nuherds()} ⇒ integer-and-list [\#]
]
\par nuherds() returns the number of unique herds in the pedigree along with a list of the herds.
\begin{description}
\item[\textit{self}
]
Reference to the current Pedigree() object
\item[Returns:
]
The number of unique herds in the pedigree and a list of those herds
\end{description}\\

\item[\textbf{nus()} ⇒ integer-and-list [\#]
]
\par nus() returns the number of unique sires in the pedigree along with a list of the sires
\begin{description}
\item[\textit{self}
]
Reference to the current Pedigree() object
\item[Returns:
]
The number of unique sires in the pedigree and a list of those sires
\end{description}\\

\item[\textbf{nuy()} ⇒ integer-and-list [\#]
]
\par nuy() returns the number of unique birthyears in the pedigree along with a list of the birthyears
\begin{description}
\item[\textit{self}
]
Reference to the current Pedigree() object
\item[Returns:
]
The number of unique birthyears in the pedigree and a list of those birthyears
\end{description}\\

\item[\textbf{printme()} [\#]
]
\par printme() prints a summary of the metadata stored in the Pedigree() object.
\begin{description}
\item[\textit{self}
]
Reference to the current Pedigree() object
\end{description}\\

\item[\textbf{stringme()} [\#]
]
\par stringme() returns a summary of the metadata stored in the pedigree as
a string.
\begin{description}
\item[\textit{self}
]
Reference to the current Pedigree() object
\end{description}\\

\end{description}
\subsection*{The PyPedalError Class}
\begin{description}
\item[\textbf{PyPedalError} (class)  [\#]
]
\par PyPedalError is the base class for exceptions in PyPedal. The exceptions
are based on the examples from "An Introduction to Python" by Guido van
Rossum and Fred L. Drake,Jr.
(http://www.network-theory.co.uk/docs/pytut/tut\_64.html).

\end{description}
\subsection*{The PyPedalPedigreeInputFileNameError Class}
\begin{description}
\item[\textbf{PyPedalPedigreeInputFileNameError()} (class) [\#]
]
\par PyPedalPedigreeInputFileNameError is raised when a simulated pedigree
is not requested and a pedigree file name is not provided.

\end{description}
\subsection*{The SimAnimal Class}
\begin{description}
\item[\textbf{SimAnimal(animalID, sireID=0, damID=0, sex='u', gen=0)} (class) [\#]
]
\par The SimAnimal() class is a placeholder used for simulating animals.

\item[\textbf{\_\_init\_\_(animalID, sireID=0, damID=0, sex='u', gen=0)} ⇒ object [\#]
]
\par \_\_init\_\_() initializes a SimAnimal() object.
\begin{description}
\item[\textit{animalID}
]
Animal's ID.
\item[\textit{sireID}
]
Sire's ID.
\item[\textit{damID}
]
Dam's ID.
\item[\textit{sex}
]
Sex of animal.
\item[Returns:
]
An instance of a SimAnimal() object populated with data
\end{description}\\

\item[\textbf{printme()} ⇒ None [\#]
]
\par printme() prints a summary of the data stored in a SimAnimal() object.
\begin{description}
\item[\textit{self}
]
Reference to the current object
\item[Returns:
]
None
\end{description}\\

\item[\textbf{stringme()} ⇒ None [\#]
]
\par stringme() returns the data stored in a SimAnimal() object as a string.
\begin{description}
\item[\textit{self}
]
Reference to the current NewAnimal() object
\item[Returns:
]
None
\end{description}\\

\end{description}
\subsection*{The tail\_recursive Class}
\begin{description}
\item[\textbf{tail\_recursive(func)} (class) [\#]
]
\par tail\_recursive is a tail recursion decorator that eliminates tail calls
for recursive functions. It was taken from the Python Cookbook entry
"New Tail Recursion Decorator" submitted by Kay Schlueh. This version was
provided by Michele Simionato on 2006/05/15. The URL is:
http://aspn.activestate.com/ASPN/Cookbook/Python/Recipe/496691.

\end{description}
\end{document}
