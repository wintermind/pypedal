pyp\_newclasses\index[func]{pyp_newclasses}\label{sec:functions-pyp-newclasses} contains the new class structure that is used by PyPedal 2.0.0. It includes a master pedigree class, a NewAnimal() class, a PedigreeMetadata() class, and a NewAMatrix class.

\subsection*{Module Contents}

\begin{description}
\item[\textbf{LightAnimal(locations, data, mykw)} (class)]\index[func]{pyp_newclasses!LightAnimal}
The LightAnimal() class is holds animals records read from a pedigree file. It is a lightweight class whose objects use much less memory than \class{NewAnimal} objects.
For more information about this class, see \emph{The LightAnimal Class}

\item[\textbf{NewAMatrix(kw)} (class)]\index[func]{pyp_newclasses!NewAMatrix}
NewAMatrix provides an instance of a numerator relationship matrix as a Numarray array of floats with some convenience methods.
For more information about this class, see \emph{The NewAMatrix Class}

\item[\textbf{NewAnimal(locations, data, mykw)} (class)]\index[func]{pyp_newclasses!NewAnimal}
The NewAnimal() class is holds animals records read from a pedigree file.
For more information about this class, see \emph{The NewAnimal Class}

\item[\textbf{NewPedigree(kw)} (class)]\index[func]{pyp_newclasses!NewPedigree}
The NewPedigree class is the main data structure for PyP 2.0.0Final.
For more information about this class, see \emph{The NewPedigree Class}

\item[\textbf{PedigreeMetadata(myped, kw)} (class)]\index[func]{pyp_newclasses!PedigreeMetadata}
The PedigreeMetadata() class stores metadata about pedigrees.
For more information about this class, see \emph{The PedigreeMetadata Class}

\item[\textbf{SimAnimal(locations, data, mykw)} (class)]\index[func]{pyp_newclasses!SimAnimal}
The SimAnimal() class is a lightweight animal class that is used by the pedigree simulation routine in \function{pyp_newclasses.simulate}.
For more information about this class, see \emph{The SimAnimal Class}

\item[\textbf{loadPedigree(options='', optionsfile='pypedal.ini')} $\Rightarrow$ integer]
\index[func]{pyp_newclasses!loadPedigree()}
loadPedigree() wraps pedigree creation and loading into a one-step process.  If the user passes both a dictionary and a filename, the dictionary will be used instead of the filename unless the dictionary is empty.
\begin{description}
\item[\emph{options}] Dictionary of pedigree options.
\item[\emph{optionsfile}] File from which pedigree options should be read.
\item[Returns:] An instance of a \class{NewPedigree} object on success, a 0 otherwise.
\end{description}

\end{description}

\subsection*{The LightAnimal Class}
\begin{description}
\item[\textbf{LightAnimal(locations, data, mykw)} (class)]
The LightAnimal() class holds animals records read from a pedigree file. It is a much simpler object than the \class{NewAnimal} object and is intended for use with the graph theoretic routines in \module{pyp\_network}. The only attributes of these objects are: animal ID, sire ID, dam ID, original ID, birth year, and sex.

\item[\textbf{\_\_init\_\_(locations, data, mykw)} $\Rightarrow$ object]\index[func]{pyp_newclasses!LightAnimal!\_\_init\_\_()}
\_\_init\_\_() initializes a LightAnimal() object.
\begin{description}
\item[\emph{locations}] A dictionary containing the locations of variables in the input line.
\item[\emph{data}] The line of input read from the pedigree file.
\item[Returns:] An instance of a LightAnimal() object populated with data
\end{description}

\item[\textbf{dictme()} $\Rightarrow$]\index[func]{pyp_newclasses!LightAnimal!dictme()}
dictme() returns a summary of the data stored in the LightAnimal() object as a dictionary.
\begin{description}
\item[\emph{self}] Reference to the current LightAnimal() object
\end{description}

\item[\textbf{pad\_id()} $\Rightarrow$ integer]\index[func]{pyp_newclasses!LightAnimal!pad_id()}
pad\_id() takes an Animal ID, pads it to fifteen digits, and prepends the birthyear (or 1950 if the birth year is unknown). The order of elements is: birthyear, animalID, count of zeros, zeros.
\begin{description}
\item[\emph{self}] Reference to the current LightAnimal() object
\item[Returns:] A padded ID number that is supposed to be unique across animals
\end{description}

\item[\textbf{printme()} $\Rightarrow$ None]\index[func]{pyp_newclasses!LightAnimal!printme()}
printme() prints a summary of the data stored in the LightAnimal() object.
\begin{description}
\item[\emph{self}] Reference to the current LightAnimal() object
\end{description}

\item[\textbf{string\_to\_int(idstring)} $\Rightarrow$ None]\index[func]{pyp_newclasses!LightAnimal!string_to_int()}
string\_to\_int() takes an Animal/Sire/Dam ID as a string and returns a string that can be represented as an integer by replacing each character in the string with its corresponding ASCII table value.

\item[\textbf{stringme()} $\Rightarrow$ None]\index[func]{pyp_newclasses!LightAnimal!stringme()}
stringme() returns a summary of the data stored in the LightAnimal() object as a string.
\begin{description}
\item[\emph{self}] Reference to the current LightAnimal() object
\end{description}

\item[\textbf{trap()} $\Rightarrow$ None]\index[func]{pyp_newclasses!LightAnimal!trap()}
trap() checks for common errors in LightAnimal() objects
\begin{description}
\item[\emph{self}] Reference to the current LightAnimal() object
\end{description}

\end{description}

\subsection*{The NewAMatrix Class}
\begin{description}
\item[\textbf{NewAMatrix(kw)} (class)]
NewAMatrix provides an instance of a numerator relationship matrix as a Numarray array of floats with some convenience methods. The idea here is to provide a wrapper around a NRM so that it is easier to work with. For large pedigrees it can take a long time to compute the elements of A, so there is real value in providing an easy way to save and retrieve a NRM once it has been formed.

\item[\textbf{\_\_init\_\_(self, kw)} $\Rightarrow$ object]\index[func]{pyp_newclasses!NewAMatrix!\_\_init\_\_()}
\_\_init\_\_() initializes a NewAMatrix() object.
\begin{description}
\item[\emph{kw}] A dictionary of options.
\item[Returns:] An instance of a NewAMatrix() object populated with data
\end{description}

\item[\textbf{form\_a\_matrix(pedigree)} $\Rightarrow$ integer]\index[func]{pyp_newclasses!NewAMatrix!form_a_matrix()}
form\_a\_matrix() calls pyp\_nrm/fast\_a\_matrix() or pyp\_nrm/fast\_a\_matrix\_r() to form a NRM from a pedigree.
\begin{description}
\item[\emph{pedigree}] The pedigree used to form the NRM.
\item[Returns:] A NRM on success, 0 on failure.
\end{description}

\item[\textbf{info()} $\Rightarrow$ None]\index[func]{pyp_newclasses!NewAMatrix!info()}
info() uses the info() method of Numarray arrays to dump some information about the NRM. This is of use predominantly for debugging.
\begin{description}
\item[\emph{None}]
\item[Returns:] None
\end{description}

\item[\textbf{load(nrm\_filename)} $\Rightarrow$ integer]\index[func]{pyp_newclasses!NewAMatrix!load()}
load() uses the Numarray Array Function ``fromfile()'' to load an array from a binary file. If the load is successful, self.nrm contains the matrix.
\begin{description}
\item[\emph{nrm\_filename}] The file from which the matrix should be read.
\item[Returns:] A load status indicator (0: failed, 1: success).
\end{description}

\item[\textbf{printme()} $\Rightarrow$ None]\index[func]{pyp_newclasses!NewAMatrix!printme()}
printme() prints the NRM to STDOUT.
\begin{description}
\item[\emph{self}] Reference to the current NewAMatrix() object
\end{description}

\item[\textbf{save(nrm\_filename)} $\Rightarrow$ integer]\index[func]{pyp_newclasses!NewAMatrix!save()}
save() uses the Numarray method ``tofile()'' to save an array to a binary file.
\begin{description}
\item[\emph{nrm\_filename}] The file to which the matrix should be written.
\item[Returns:] A save status indicator (0: failed, 1: success).
\end{description}

\end{description}

\subsection*{The NewAnimal Class}
\begin{description}
\item[\textbf{NewAnimal(locations, data, mykw)} (class)]
The NewAnimal() class is holds animals records read from a pedigree file.

\item[\textbf{\_\_init\_\_(locations, data, mykw)} $\Rightarrow$ object]\index[func]{pyp_newclasses!NewAnimal!\_\_init\_\_()}
\_\_init\_\_() initializes a NewAnimal() object.
\begin{description}
\item[\emph{locations}] A dictionary containing the locations of variables in the input line.
\item[\emph{data}] The line of input read from the pedigree file.
\item[Returns:] An instance of a NewAnimal() object populated with data
\end{description}

\item[\textbf{dictme()} $\Rightarrow$]\index[func]{pyp_newclasses!NewAnimal!dictme()}
dictme() returns a summary of the data stored in the NewAnimal() object as a dictionary.
\begin{description}
\item[\emph{self}] Reference to the current LightAnimal() object
\end{description}

\item[\textbf{pad\_id()} $\Rightarrow$ integer]\index[func]{pyp_newclasses!NewAnimal!pad_id()}
pad\_id() takes an Animal ID, pads it to fifteen digits, and prepends the birthyear (or 1950 if the birth year is unknown). The order of elements is: birthyear, animalID, count of zeros, zeros.
\begin{description}
\item[\emph{self}] Reference to the current Animal() object
\item[Returns:] A padded ID number that is supposed to be unique across animals
\end{description}

\item[\textbf{printme()} $\Rightarrow$ None]\index[func]{pyp_newclasses!NewAnimal!printme()}
printme() prints a summary of the data stored in the Animal() object.
\begin{description}
\item[\emph{self}] Reference to the current Animal() object
\end{description}


\item[\textbf{string\_to\_int(idstring)} $\Rightarrow$ None]\index[func]{pyp_newclasses!NewAnimal!string_to_int()}
string\_to\_int() takes an Animal/Sire/Dam ID as a string and returns a string that can be represented as an integer by replacing each character in the string with its corresponding ASCII table value.

\item[\textbf{stringme()} $\Rightarrow$ None]\index[func]{pyp_newclasses!NewAnimal!stringme()}
stringme() returns a summary of the data stored in the Animal() object as a string.
\begin{description}
\item[\emph{self}] Reference to the current Animal() object
\end{description}


\item[\textbf{trap()} $\Rightarrow$ None]\index[func]{pyp_newclasses!NewAnimal!trap()}
trap() checks for common errors in Animal() objects
\begin{description}
\item[\emph{self}] Reference to the current Animal() object
\end{description}

\end{description}

\subsection*{The NewPedigree Class}
\begin{description}
\item[\textbf{NewPedigree(kw)} (class)]
The NewPedigree class is the main data structure for PyP 2.0.0Final.

\item[\textbf{\_\_init\_\_(self, kw)} $\Rightarrow$ object]\index[func]{pyp_newclasses!NewPedigree!\_\_init\_\_()}
\_\_init\_\_() initializes a NewPedigree() object.
\begin{description}
\item[\emph{kw}] A dictionary of options.
\item[Returns:] An instance of a NewPedigree() object populated with data
\end{description}

\item[\textbf{addanimal(animalID, sireID, damID)} $\Rightarrow$ integer]\index[func]{pyp_newclasses!NewPedigree!addanimal()}
addanimal() adds a new animal of class NewAnimal to the pedigree.
\begin{description}
\item[Returns:] 1 on success, 0 on failure
\end{description}

\item[\textbf{delanimal(animalID)} $\Rightarrow$ integer]\index[func]{pyp_newclasses!NewPedigree!delanimal()}
delanimal() deletes an animal from the pedigree. Note that this method DOES not update the metadata attached to the pedigree and should only be used if that is not important. As of 04/10/2006 delanimal() is intended for use by \function{pyp\_metrics.mating\_coi()} rather than directly by users.
\begin{description}
\item[\emph{animalID}] ID of the animal to be deleted
\item[Returns:] 1 on success, 0 on failure
\end{description}

\item[\textbf{fromgraph(pedgraph)} $\Rightarrow$ None]\index[func]{pyp_newclasses!NewPedigree!fromgraph()}
fromgraph() loads the animals to populate the pedigree from an XDiGraph project.
\begin{description}
\item[\emph{pedgraph}] NetworkX object from which the pedigree should be created.
\item[Returns:] None
\end{description}

\item[\textbf{load(pedsource='file')} $\Rightarrow$ None]\index[func]{pyp_newclasses!NewPedigree!load()}
load() wraps several processes useful for loading and preparing a pedigree for use in an analysis, including reading the animals into a list of animal objects, forming lists of sires and dams, checking for common errors, setting ancestor flags, and renumbering the pedigree.
\begin{description}
\item[\emph{renum}] Flag to indicate whether or not the pedigree is to be renumbered.
\item[\emph{alleles}] Flag to indicate whether or not pyp\_metrics/effective\_founder\_genomes() should be called for a single round to assign alleles.
\item[Returns:] None
\end{description}

\item[\textbf{preprocess()} $\Rightarrow$ None]\index[func]{pyp_newclasses!NewPedigree!preprocess()}
preprocess() processes a pedigree file, which includes reading the animals into a list of animal objects, forming lists of sires and dams, and checking for common errors.
\begin{description}
\item[\emph{None}]
\item[Returns:] None
\end{description}

\item[\textbf{printoptions()} $\Rightarrow$ None]\index[func]{pyp_newclasses!NewPedigree!printoptions()}
printoptions() prints the contents of a pedigree's \member{kw} dictionary.
\begin{description}
\item[\emph{None}]
\item[Returns:] None
\end{description}

\item[\textbf{renumber()} $\Rightarrow$ None]\index[func]{pyp_newclasses!NewPedigree!renumber()}
renumber() updates the ID map after a pedigree has been renumbered so that all references are to renumbered rather than original IDs.
\begin{description}
\item[\emph{None}]
\item[Returns:] None
\end{description}

\item[\textbf{save(filename='', outformat='o', idformat='o')} $\Rightarrow$ integer]\index[func]{pyp_newclasses!NewPedigree!save()}
save() writes a PyPedal pedigree to a user-specified file. The saved pedigree includes all fields recognized by PyPedal, not just the original fields read from the input pedigree file.
\begin{description}
\item[\emph{filename}] The file to which the pedigree should be written.
\item[\emph{outformat}] The format in which the pedigree should be written: 'o' for original (as read) and 'l' for long version (all available variables).
\item[\emph{idformat}] Write 'o' (original) or 'r' (renumbered) animal, sire, and dam IDs.
\item[Returns:] A save status indicator (0: failed, 1: success)
\end{description}

\item[\textbf{simulate()} $\Rightarrow$ None]\index[func]{pyp_newclasses!NewPedigree!simulate()}
simulate() simulates an arbitrary pedigree of size n with g generations starting from n\_s base sires and n\_d base dams. This method is based on the concepts and algorithms in the Pedigree::sample method from Matvec 1.1a. The arguments are read from the pedigree object's options dictionary. The options used to control pedigree simulation are presented in Table \ref{tbl:options}, and detailed commentary is provided in section \ref{sec:pedigree-simulation}.
\begin{description}
\item[\emph{None}]
\item[Returns:] None
\end{description}

\item[\textbf{updateidmap()} $\Rightarrow$ None]\index[func]{pyp_newclasses!NewPedigree!updateidmap()}
updateidmap() updates the ID map after a pedigree has been renumbered so that all references are to renumbered rather than original IDs.
\begin{description}
\item[\emph{None}]
\item[Returns:] None
\end{description}

\end{description}

\subsection*{The PedigreeMetadata Class}
\begin{description}
\item[\textbf{PedigreeMetadata(myped, kw)} (class)]
The PedigreeMetadata() class stores metadata about pedigrees. Hopefully this will help improve performance in some procedures, as well as provide some useful summary data.

\item[\textbf{\_\_init\_\_(myped, kw)} $\Rightarrow$ object]\index[func]{pyp_newclasses!PedigreeMetadata!\_\_init\_\_()}
\_\_init\_\_() initializes a PedigreeMetadata object.
\begin{description}
\item[\emph{self}] Reference to the current Pedigree() object
\item[\emph{myped}] A PyPedal pedigree.
\item[\emph{kw}] A dictionary of options.
\item[Returns:] An instance of a Pedigree() object populated with data
\end{description}

\item[\textbf{fileme()} $\Rightarrow$ None]\index[func]{pyp_newclasses!PedigreeMetadata!fileme()}
fileme() writes the metada stored in the Pedigree() object to disc.
\begin{description}
\item[\emph{self}] Reference to the current Pedigree() object
\end{description}

\item[\textbf{nud()} $\Rightarrow$ integer-and-list]\index[func]{pyp_newclasses!PedigreeMetadata!nud()}
nud() returns the number of unique dams in the pedigree along with a list of the dams
\begin{description}
\item[\emph{self}] Reference to the current Pedigree() object
\item[Returns:] The number of unique dams in the pedigree and a list of those dams
\end{description}

\item[\textbf{nuf()} $\Rightarrow$ integer-and-list]\index[func]{pyp_newclasses!PedigreeMetadata!nuf()}
nuf() returns the number of unique founders in the pedigree along with a list of the founders
\begin{description}
\item[\emph{self}] Reference to the current Pedigree() object
\item[Returns:] The number of unique founders in the pedigree and a list of those founders
\end{description}

\item[\textbf{nug()} $\Rightarrow$ integer-and-list]\index[func]{pyp_newclasses!PedigreeMetadata!nug()}
nug() returns the number of unique generations in the pedigree along with a list of the generations
\begin{description}
\item[\emph{self}] Reference to the current Pedigree() object
\item[Returns:] The number of unique generations in the pedigree and a list of those generations
\end{description}

\item[\textbf{nuherds()} $\Rightarrow$ integer-and-list]\index[func]{pyp_newclasses!PedigreeMetadata!nuherds()}
nuherds() returns the number of unique herds in the pedigree along with a list of the herdss
\begin{description}
\item[\emph{self}] Reference to the current Pedigree() object
\item[Returns:] The number of unique herds in the pedigree and a list of those herds
\end{description}

\item[\textbf{nus()} $\Rightarrow$ integer-and-list]\index[func]{pyp_newclasses!PedigreeMetadata!nus()}
nus() returns the number of unique sires in the pedigree along with a list of the sires
\begin{description}
\item[\emph{self}] Reference to the current Pedigree() object
\item[Returns:] The number of unique sires in the pedigree and a list of those sires
\end{description}

\item[\textbf{nuy()} $\Rightarrow$ integer-and-list]\index[func]{pyp_newclasses!PedigreeMetadata!nuy()}
nuy() returns the number of unique birthyears in the pedigree along with a list of the birthyears
\begin{description}
\item[\emph{self}] Reference to the current Pedigree() object
\item[Returns:] The number of unique birthyears in the pedigree and a list of those birthyears
\end{description}

\item[\textbf{printme()} $\Rightarrow$ None]\index[func]{pyp_newclasses!PedigreeMetadata!printme()}
printme() prints a summary of the metadata stored in the Pedigree() object.
\begin{description}
\item[\emph{self}] Reference to the current Pedigree() object
\end{description}

\item[\textbf{stringme()} $\Rightarrow$ None]\index[func]{pyp_newclasses!PedigreeMetadata!stringme()}
stringme() returns a summary of the metadata stored in the pedigree as a string.
\begin{description}
\item[\emph{self}] Reference to the current Pedigree() object
\end{description}

\end{description}

\subsection*{The PyPedalError Class}
\begin{description}
\item[\textbf{PyPedalError} (class)]
PyPedalError is the base class for exceptions in PyPedal. The exceptions are based on the examples from ``An Introduction to Python'' by Guido van Rossum and Fred L. Drake,Jr. (\url{http://www.network-theory.co.uk/docs/pytut/tut}\_64.html).
\end{description}

\subsection*{The PyPedalPedigreeInputFileNameError Class}
\begin{description}
\item[\textbf{PyPedalPedigreeInputFileNameError()} (class)]
PyPedalPedigreeInputFileNameError is raised when a simulated pedigree is not requested and a pedigree file name is not provided.
\end{description}

\subsection*{The SimAnimal Class}
\begin{description}
\item[\textbf{SimAnimal(animalID,sireID,damID,sex,gen)} (class)]
SimAnimal() is a very lightweight class used for simulating pedigrees.

\item[\textbf{\_\_init\_\_(animalID,sireID=0,damID=0,sex='u',gen=0)} $\Rightarrow$ object]\index[func]{pyp_newclasses!SimAnimal!\_\_init\_\_()}
\_\_init\_\_() initializes a SimAnimal() object.
\begin{description}
\item[\emph{animalID}] The animal's ID
\item[\emph{sireID}] The sire's ID
\item[\emph{damID}] The dam's ID
\item[\emph{sex}] The sex of the animal ('m'|'f'|'u')
\item[\emph{gen}] The animal's generation code.
\item[Returns:] An instance of a SimAnimal() object populated with data
\end{description}

\item[\textbf{printme()} $\Rightarrow$ None]\index[func]{pyp_newclasses!SimAnimal!printme()}
printme() prints a summary of the data stored in the SimAnimal() object.
\begin{description}
\item[\emph{self}] Reference to the current SimAnimal() object
\item[Returns:] None
\end{description}

\item[\textbf{stringme()} $\Rightarrow$ None]\index[func]{pyp_newclasses!SimAnimal!stringme()}
stringme() returns a summary of the data stored in the SimAnimal() object as a string.
\begin{description}
\item[\emph{self}] Reference to the current SimAnimal() object
\item[Returns:] A string containing the data in the SimAnimal() object
\end{description}

\end{description}